% Options for packages loaded elsewhere
\PassOptionsToPackage{unicode}{hyperref}
\PassOptionsToPackage{hyphens}{url}
%
\documentclass[
]{article}
\usepackage{amsmath,amssymb}
\usepackage{iftex}
\ifPDFTeX
  \usepackage[T1]{fontenc}
  \usepackage[utf8]{inputenc}
  \usepackage{textcomp} % provide euro and other symbols
\else % if luatex or xetex
  \usepackage{unicode-math} % this also loads fontspec
  \defaultfontfeatures{Scale=MatchLowercase}
  \defaultfontfeatures[\rmfamily]{Ligatures=TeX,Scale=1}
\fi
\usepackage{lmodern}
\ifPDFTeX\else
  % xetex/luatex font selection
\fi
% Use upquote if available, for straight quotes in verbatim environments
\IfFileExists{upquote.sty}{\usepackage{upquote}}{}
\IfFileExists{microtype.sty}{% use microtype if available
  \usepackage[]{microtype}
  \UseMicrotypeSet[protrusion]{basicmath} % disable protrusion for tt fonts
}{}
\makeatletter
\@ifundefined{KOMAClassName}{% if non-KOMA class
  \IfFileExists{parskip.sty}{%
    \usepackage{parskip}
  }{% else
    \setlength{\parindent}{0pt}
    \setlength{\parskip}{6pt plus 2pt minus 1pt}}
}{% if KOMA class
  \KOMAoptions{parskip=half}}
\makeatother
\usepackage{xcolor}
\usepackage[margin=1in]{geometry}
\usepackage{color}
\usepackage{fancyvrb}
\newcommand{\VerbBar}{|}
\newcommand{\VERB}{\Verb[commandchars=\\\{\}]}
\DefineVerbatimEnvironment{Highlighting}{Verbatim}{commandchars=\\\{\}}
% Add ',fontsize=\small' for more characters per line
\usepackage{framed}
\definecolor{shadecolor}{RGB}{248,248,248}
\newenvironment{Shaded}{\begin{snugshade}}{\end{snugshade}}
\newcommand{\AlertTok}[1]{\textcolor[rgb]{0.94,0.16,0.16}{#1}}
\newcommand{\AnnotationTok}[1]{\textcolor[rgb]{0.56,0.35,0.01}{\textbf{\textit{#1}}}}
\newcommand{\AttributeTok}[1]{\textcolor[rgb]{0.13,0.29,0.53}{#1}}
\newcommand{\BaseNTok}[1]{\textcolor[rgb]{0.00,0.00,0.81}{#1}}
\newcommand{\BuiltInTok}[1]{#1}
\newcommand{\CharTok}[1]{\textcolor[rgb]{0.31,0.60,0.02}{#1}}
\newcommand{\CommentTok}[1]{\textcolor[rgb]{0.56,0.35,0.01}{\textit{#1}}}
\newcommand{\CommentVarTok}[1]{\textcolor[rgb]{0.56,0.35,0.01}{\textbf{\textit{#1}}}}
\newcommand{\ConstantTok}[1]{\textcolor[rgb]{0.56,0.35,0.01}{#1}}
\newcommand{\ControlFlowTok}[1]{\textcolor[rgb]{0.13,0.29,0.53}{\textbf{#1}}}
\newcommand{\DataTypeTok}[1]{\textcolor[rgb]{0.13,0.29,0.53}{#1}}
\newcommand{\DecValTok}[1]{\textcolor[rgb]{0.00,0.00,0.81}{#1}}
\newcommand{\DocumentationTok}[1]{\textcolor[rgb]{0.56,0.35,0.01}{\textbf{\textit{#1}}}}
\newcommand{\ErrorTok}[1]{\textcolor[rgb]{0.64,0.00,0.00}{\textbf{#1}}}
\newcommand{\ExtensionTok}[1]{#1}
\newcommand{\FloatTok}[1]{\textcolor[rgb]{0.00,0.00,0.81}{#1}}
\newcommand{\FunctionTok}[1]{\textcolor[rgb]{0.13,0.29,0.53}{\textbf{#1}}}
\newcommand{\ImportTok}[1]{#1}
\newcommand{\InformationTok}[1]{\textcolor[rgb]{0.56,0.35,0.01}{\textbf{\textit{#1}}}}
\newcommand{\KeywordTok}[1]{\textcolor[rgb]{0.13,0.29,0.53}{\textbf{#1}}}
\newcommand{\NormalTok}[1]{#1}
\newcommand{\OperatorTok}[1]{\textcolor[rgb]{0.81,0.36,0.00}{\textbf{#1}}}
\newcommand{\OtherTok}[1]{\textcolor[rgb]{0.56,0.35,0.01}{#1}}
\newcommand{\PreprocessorTok}[1]{\textcolor[rgb]{0.56,0.35,0.01}{\textit{#1}}}
\newcommand{\RegionMarkerTok}[1]{#1}
\newcommand{\SpecialCharTok}[1]{\textcolor[rgb]{0.81,0.36,0.00}{\textbf{#1}}}
\newcommand{\SpecialStringTok}[1]{\textcolor[rgb]{0.31,0.60,0.02}{#1}}
\newcommand{\StringTok}[1]{\textcolor[rgb]{0.31,0.60,0.02}{#1}}
\newcommand{\VariableTok}[1]{\textcolor[rgb]{0.00,0.00,0.00}{#1}}
\newcommand{\VerbatimStringTok}[1]{\textcolor[rgb]{0.31,0.60,0.02}{#1}}
\newcommand{\WarningTok}[1]{\textcolor[rgb]{0.56,0.35,0.01}{\textbf{\textit{#1}}}}
\usepackage{longtable,booktabs,array}
\usepackage{calc} % for calculating minipage widths
% Correct order of tables after \paragraph or \subparagraph
\usepackage{etoolbox}
\makeatletter
\patchcmd\longtable{\par}{\if@noskipsec\mbox{}\fi\par}{}{}
\makeatother
% Allow footnotes in longtable head/foot
\IfFileExists{footnotehyper.sty}{\usepackage{footnotehyper}}{\usepackage{footnote}}
\makesavenoteenv{longtable}
\usepackage{graphicx}
\makeatletter
\def\maxwidth{\ifdim\Gin@nat@width>\linewidth\linewidth\else\Gin@nat@width\fi}
\def\maxheight{\ifdim\Gin@nat@height>\textheight\textheight\else\Gin@nat@height\fi}
\makeatother
% Scale images if necessary, so that they will not overflow the page
% margins by default, and it is still possible to overwrite the defaults
% using explicit options in \includegraphics[width, height, ...]{}
\setkeys{Gin}{width=\maxwidth,height=\maxheight,keepaspectratio}
% Set default figure placement to htbp
\makeatletter
\def\fps@figure{htbp}
\makeatother
\setlength{\emergencystretch}{3em} % prevent overfull lines
\providecommand{\tightlist}{%
  \setlength{\itemsep}{0pt}\setlength{\parskip}{0pt}}
\setcounter{secnumdepth}{5}
\ifLuaTeX
  \usepackage{selnolig}  % disable illegal ligatures
\fi
\usepackage{bookmark}
\IfFileExists{xurl.sty}{\usepackage{xurl}}{} % add URL line breaks if available
\urlstyle{same}
\hypersetup{
  pdftitle={Statistics for Neuroscience -- Lab Activity 2},
  pdfauthor={Livio Finos},
  hidelinks,
  pdfcreator={LaTeX via pandoc}}

\title{Statistics for Neuroscience -- Lab Activity 2}
\author{Livio Finos}
\date{}

\begin{document}
\maketitle

{
\setcounter{tocdepth}{2}
\tableofcontents
}
\subsection{Multiple Linear Regression Homework: Model Selection via
F-tests, Adjusted R², AIC, and Effect
Sizes}\label{multiple-linear-regression-homework-model-selection-via-f-tests-adjusted-ruxb2-aic-and-effect-sizes}

\textbf{Objective}: Select and justify the best model for diabetes
progression (using in-sample criteria only): F-tests, adjusted R²,
AIC/BIC, and standardized effect sizes.

\subsubsection{Part 1: Data loading and exploration (10
points)}\label{part-1-data-loading-and-exploration-10-points}

\begin{Shaded}
\begin{Highlighting}[]
\FunctionTok{library}\NormalTok{(tidyverse)}
\end{Highlighting}
\end{Shaded}

\begin{verbatim}
## -- Attaching core tidyverse packages ------------------------ tidyverse 2.0.0 --
## v dplyr     1.1.4     v readr     2.1.5
## v forcats   1.0.1     v stringr   1.5.2
## v ggplot2   4.0.0     v tibble    3.3.0
## v lubridate 1.9.4     v tidyr     1.3.1
## v purrr     1.1.0     
## -- Conflicts ------------------------------------------ tidyverse_conflicts() --
## x dplyr::filter() masks stats::filter()
## x dplyr::lag()    masks stats::lag()
## i Use the conflicted package (<http://conflicted.r-lib.org/>) to force all conflicts to become errors
\end{verbatim}

\begin{Shaded}
\begin{Highlighting}[]
\NormalTok{diab }\OtherTok{\textless{}{-}} \FunctionTok{read\_csv}\NormalTok{(}\StringTok{"./data/diabetes\_sklearn.csv"}\NormalTok{)}
\end{Highlighting}
\end{Shaded}

\begin{verbatim}
## Rows: 442 Columns: 11
## -- Column specification --------------------------------------------------------
## Delimiter: ","
## dbl (11): age, sex, bmi, bp, s1, s2, s3, s4, s5, s6, target
## 
## i Use `spec()` to retrieve the full column specification for this data.
## i Specify the column types or set `show_col_types = FALSE` to quiet this message.
\end{verbatim}

\begin{longtable}[]{@{}
  >{\raggedright\arraybackslash}p{(\columnwidth - 6\tabcolsep) * \real{0.0645}}
  >{\raggedright\arraybackslash}p{(\columnwidth - 6\tabcolsep) * \real{0.2823}}
  >{\raggedright\arraybackslash}p{(\columnwidth - 6\tabcolsep) * \real{0.2177}}
  >{\raggedright\arraybackslash}p{(\columnwidth - 6\tabcolsep) * \real{0.4355}}@{}}
\toprule\noalign{}
\begin{minipage}[b]{\linewidth}\raggedright
Variable
\end{minipage} & \begin{minipage}[b]{\linewidth}\raggedright
Description
\end{minipage} & \begin{minipage}[b]{\linewidth}\raggedright
Units (pre-standardization)
\end{minipage} & \begin{minipage}[b]{\linewidth}\raggedright
Notes
\end{minipage} \\
\midrule\noalign{}
\endhead
\bottomrule\noalign{}
\endlastfoot
age & Age & Years & - \\
sex & Sex & M/F coded quantitatively & Typically -0.0446 (female),
+0.0507 (male) \\
bmi & Body mass index & kg/m² & Strongest univariate predictor \\
bp & Average blood pressure & mm Hg & - \\
s1 & Total serum cholesterol (tc) & mg/dL & - \\
s2 & Low-density lipoprotein (ldl) & mg/dL & ``Bad'' cholesterol \\
s3 & High-density lipoprotein (hdl) & mg/dL & ``Good'' cholesterol \\
s4 & Total cholesterol / HDL ratio (tch) & Ratio & Higher = worse risk
profile \\
s5 & Log serum triglycerides (ltg) & log(mg/dL) & Often second-strongest
predictor \\
s6 & Blood glucose level (glu) & mg/dL & Fasting blood sugar \\
target & Disease progression & Quantitative score & One year after
baseline; mean=152, SD=77, range 25--346 \\
\end{longtable}

\textbf{Tasks}:

\begin{enumerate}
\def\labelenumi{\arabic{enumi}.}
\tightlist
\item
  Report predictor correlations with \texttt{target}, and pairwise
  correlations among top 3 predictors of \texttt{target}.\\
\item
  Scatterplot matrix of \texttt{target} vs \texttt{\{bmi,\ s5,\ bp\}}.
  Comment on linearity.
\end{enumerate}

\subsubsection{Part 2: Baseline models (20
points)}\label{part-2-baseline-models-20-points}

Fit these nested models:

\begin{Shaded}
\begin{Highlighting}[]
\NormalTok{m\_null  }\OtherTok{\textless{}{-}} \FunctionTok{lm}\NormalTok{(target }\SpecialCharTok{\textasciitilde{}} \DecValTok{1}\NormalTok{, }\AttributeTok{data =}\NormalTok{ diab)                    }\CommentTok{\# Intercept only}
\NormalTok{m\_small }\OtherTok{\textless{}{-}} \FunctionTok{lm}\NormalTok{(target }\SpecialCharTok{\textasciitilde{}}\NormalTok{ age }\SpecialCharTok{+}\NormalTok{ sex }\SpecialCharTok{+}\NormalTok{ bmi }\SpecialCharTok{+}\NormalTok{ bp, }\AttributeTok{data =}\NormalTok{ diab) }\CommentTok{\# 4 clinical basics}
\NormalTok{m\_full  }\OtherTok{\textless{}{-}} \FunctionTok{lm}\NormalTok{(target }\SpecialCharTok{\textasciitilde{}}\NormalTok{ ., }\AttributeTok{data =}\NormalTok{ diab)                    }\CommentTok{\# All 10 predictors}
\end{Highlighting}
\end{Shaded}

\textbf{Tasks}:

\begin{enumerate}
\def\labelenumi{\arabic{enumi}.}
\tightlist
\item
  Compute and tabulate adjusted R² and AIC/BIC for all three.\\
\item
  \textbf{Global F-tests}: \texttt{anova(m\_null,\ m\_small)} and
  \texttt{anova(m\_small,\ m\_full)}. Interpret p-values: evidence
  against null model? Incremental value of extra 6 predictors?\\
\item
  \textbf{Effect size}: Report partial η² for each predictor in
  \texttt{m\_full} (use \texttt{performance::model\_performance()} or
  \texttt{car::Anova()}).
\end{enumerate}

\subsubsection{Part 3: Your model selection (20
points)}\label{part-3-your-model-selection-20-points}

\textbf{Propose \texttt{m\_yours}} (4--8 predictors). Justify based
on:\\
- Domain (e.g., BMI central to diabetes).\\
- Univariate correlations.\\
- Expected confounding (e.g., control for age/sex).

Example:

\begin{Shaded}
\begin{Highlighting}[]
\NormalTok{m\_yours }\OtherTok{\textless{}{-}} \FunctionTok{lm}\NormalTok{(target }\SpecialCharTok{\textasciitilde{}}\NormalTok{ bmi }\SpecialCharTok{+}\NormalTok{ s5 }\SpecialCharTok{+}\NormalTok{ s6 }\SpecialCharTok{+}\NormalTok{ sex }\SpecialCharTok{+}\NormalTok{ age, }\AttributeTok{data =}\NormalTok{ diab)}
\end{Highlighting}
\end{Shaded}

\textbf{Tasks}:

\begin{enumerate}
\def\labelenumi{\arabic{enumi}.}
\tightlist
\item
  Tabulate adjusted R², AIC/BIC vs \texttt{m\_small} and
  \texttt{m\_full}. Which ``wins''?\\
\item
  \textbf{Hierarchical F-test}: \texttt{anova(m\_yours,\ m\_full)}.
  Significant loss in fit?\\
\item
  \textbf{Standardized effects}: Scale predictors and refit. Rank top 3
  coefficients by magnitude (effect size).
\end{enumerate}

\subsubsection{Part 4: Model comparison table (15
points)}\label{part-4-model-comparison-table-15-points}

Create this table (using \texttt{broom::glance()} +
\texttt{bind\_rows()}):

\begin{longtable}[]{@{}llllll@{}}
\toprule\noalign{}
Model & adj.R² & Δadj.R² & AIC & ΔAIC & Global F (p) \\
\midrule\noalign{}
\endhead
\bottomrule\noalign{}
\endlastfoot
m\_null & & - & & - & - \\
m\_small & & & & & \\
m\_yours & & & & & \\
m\_full & & & & & \\
\end{longtable}

\textbf{Discussion} (3--4 sentences): Best model by each criterion? Why
AIC/BIC favor parsimony? Trade-off with adj.R²?

\subsubsection{Part 5: Interpretation and effect sizes (20
points)}\label{part-5-interpretation-and-effect-sizes-20-points}

For your best model:

\begin{Shaded}
\begin{Highlighting}[]
\FunctionTok{library}\NormalTok{(car)}
\end{Highlighting}
\end{Shaded}

\begin{verbatim}
## Caricamento del pacchetto richiesto: carData
\end{verbatim}

\begin{verbatim}
## 
## Caricamento pacchetto: 'car'
\end{verbatim}

\begin{verbatim}
## Il seguente oggetto è mascherato da 'package:dplyr':
## 
##     recode
\end{verbatim}

\begin{verbatim}
## Il seguente oggetto è mascherato da 'package:purrr':
## 
##     some
\end{verbatim}

\begin{Shaded}
\begin{Highlighting}[]
\FunctionTok{Anova}\NormalTok{(m\_yours, }\AttributeTok{type=}\StringTok{"III"}\NormalTok{)  }\CommentTok{\# Type III SS for effect sizes}
\end{Highlighting}
\end{Shaded}

\begin{verbatim}
## Anova Table (Type III tests)
## 
## Response: target
##               Sum Sq  Df   F value  Pr(>F)    
## (Intercept) 10229912   1 3184.9720 < 2e-16 ***
## bmi           322025   1  100.2590 < 2e-16 ***
## s5            235764   1   73.4024 < 2e-16 ***
## s6              8786   1    2.7356 0.09886 .  
## sex             9482   1    2.9522 0.08647 .  
## age               64   1    0.0199 0.88799    
## Residuals    1400402 436                      
## ---
## Signif. codes:  0 '***' 0.001 '**' 0.01 '*' 0.05 '.' 0.1 ' ' 1
\end{verbatim}

\textbf{Tasks}: 1. Identify 2 strongest predictors (Type III F or
partial η² \textgreater{} 0.02).\\
2. Interpret: ``The effect of BMI is {[}X{]} units per SD,
p={[}Y{]}.''\\
3. \textbf{Discussion}: Confounding example? (Compare univariate vs
multivariate coefficient for BMI.)

\subsubsection{Part 6: Diagnostics (10
points)}\label{part-6-diagnostics-10-points}

\begin{Shaded}
\begin{Highlighting}[]
\FunctionTok{par}\NormalTok{(}\AttributeTok{mfrow=}\FunctionTok{c}\NormalTok{(}\DecValTok{2}\NormalTok{,}\DecValTok{2}\NormalTok{)); }\FunctionTok{plot}\NormalTok{(m\_yours)}
\end{Highlighting}
\end{Shaded}

\includegraphics{student_activity2_files/figure-latex/unnamed-chunk-5-1.pdf}

Comment briefly: violations? Influential cases?

\subsubsection{Part 7: Peer review interaction (5
points)}\label{part-7-peer-review-interaction-5-points}

\textbf{In class}: Share your table and best model. Groups vote: most
parsimonious? Most explanatory? Submit 1-sentence peer feedback.

\textbf{Total: 100 points}.

\end{document}
